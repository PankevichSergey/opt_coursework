\documentclass[a4paper,12pt]{extarticle}
\usepackage{geometry}
\usepackage[T1]{fontenc}
\usepackage[utf8]{inputenc}
\usepackage[english,russian]{babel}
\usepackage{amsmath}
\usepackage{amsthm}
\usepackage{amssymb}
\usepackage{fancyhdr}
\usepackage{setspace}
\usepackage{graphicx}
\usepackage{colortbl}
\usepackage{tikz}
\usepackage{pgf}
\usepackage{subcaption}
\usepackage{listings}
\usepackage{indentfirst}
\usepackage{algorithm}

\newcommand{\argmin}{\mathop{\rm argmin}}
\usepackage{algorithmic}
\usepackage[
backend=biber,
style=numeric,
maxbibnames=99
]{biblatex}
\addbibresource{refs.bib}
\usepackage[colorlinks,citecolor=blue,linkcolor=blue,bookmarks=false,hypertexnames=true, urlcolor=blue]{hyperref} 
\usepackage{indentfirst}
\usepackage{mathtools}
\usepackage{booktabs}
\usepackage[flushleft]{threeparttable}
\usepackage{tablefootnote}

\usepackage{chngcntr} % нумерация графиков и таблиц по секциям
\counterwithin{table}{section}
\counterwithin{figure}{section}

\graphicspath{{graphics/}}%путь к рисункам

\makeatletter
% \renewcommand{\@biblabel}[1]{#1.} % Заменяем библиографию с квадратных скобок на точку:
\makeatother

\geometry{left=2.5cm}% левое поле
\geometry{right=1.0cm}% правое поле
\geometry{top=2.0cm}% верхнее поле
\geometry{bottom=2.0cm}% нижнее поле
\setlength{\parindent}{1.25cm}
\renewcommand{\baselinestretch}{1.5} % междустрочный интервал

\newcommand{\eqdef}{\coloneqq}
\def\<#1,#2>{\langle #1,#2\rangle}
\newcommand{\norm}[1]{\|#1\|}
\newcommand{\sqn}[1]{\norm{#1}^2}
\newcommand{\bibref}[3]{\hyperlink{#1}{#2 (#3)}} % biblabel, authors, year
\addto\captionsrussian{\def\refname{Список литературы (или источников)}} 

\renewcommand{\theenumi}{\arabic{enumi}}% Меняем везде перечисления на цифра.цифра
\renewcommand{\labelenumi}{\arabic{enumi}}% Меняем везде перечисления на цифра.цифра
\renewcommand{\theenumii}{.\arabic{enumii}}% Меняем везде перечисления на цифра.цифра
\renewcommand{\labelenumii}{\arabic{enumi}.\arabic{enumii}.}% Меняем везде перечисления на цифра.цифра
\renewcommand{\theenumiii}{.\arabic{enumiii}}% Меняем везде перечисления на цифра.цифра
\renewcommand{\labelenumiii}{\arabic{enumi}.\arabic{enumii}.\arabic{enumiii}.}% Меняем везде перечисления на цифра.цифра

\begin{document}
\begin{titlepage}
\newpage

{\setstretch{1.0}
\begin{center}
ПРАВИТЕЛЬСТВО РОССИЙСКОЙ ФЕДЕРАЦИИ\\
ФГАОУ ВО НАЦИОНАЛЬНЫЙ ИССЛЕДОВАТЕЛЬСКИЙ УНИВЕРСИТЕТ\\
«ВЫСШАЯ ШКОЛА ЭКОНОМИКИ»
\\
\bigskip
Факультет компьютерных наук\\
Образовательная программа «Прикладная математика и информатика»
\end{center}
}

\vspace{2em}
УДК 519.86 % УДК нужно указывать только для исследовательсвого проекта - удалите эту строку для программного проекта
\vspace{4em}

\begin{center}
%Выберите какой у вас проект
{\bf Отчет об исследовательском проекте на тему:}\\
%{\bf Отчет о командном исследовательском проекте на тему:}\\
%{\bf Отчет о программном проекте на тему:}\\
%{\bf Отчет о командном программном проекте на тему:}\\
{\bf Распределенные алгоритмы в условиях схожести слагаемых}\\
% строчка ниже нужна только при сдача плана КР, при финальной сдаче закомментируйте ее
% (промежуточный, этап 1)
\end{center}

\vspace{2em}

{\bf Выполнил студент: \vspace{2mm}}
%{\bf Выполнили студенты: \vspace{2mm}}

{\setstretch{1.1}
\begin{tabular}{l@{\hskip 1.5cm}l}
группы \#БПМИ212, 3 курса & Панкевич Сергей Александрович \\
%группы \#БПМИ172, 3 курса & Петров Андрей Алексеевич \\
%группы \#БПМИ173, 3 курса & Иванов Андрей Алексеевич 
\end{tabular}}

% Обычно у вас есть один научный руководитель, и это человек, с которым вы работаете над проектом. Иногда по формальным причинам у вас будет руководитель (штатный сотрудник Вышки) и соруководитель (тот, с кем вы работаете), — об этом вам сообщит учебный офис (в случае с ВКР) или ЦППРиП (в случае с курсовым проектом). Также, если кто-то дополнительно вам помогал, то его можно указать как консультанта. 

%ваш официальный научник (из ВШЭ)
\vspace{1em}
{\bf Принял руководитель проекта: \vspace{2mm}}

{\setstretch{1.1}
\begin{tabular}{l}
Двинских Дарина Михайловна \\
Старший научный сотрудник\\
Факультет компьютерных наук НИУ ВШЭ 
\end{tabular}}

% со-руководитель (если есть)
%\vspace{1em}
%{\bf Соруководитель: \vspace{2mm}}%это ваш официальный научник

%{\setstretch{1.1}
%\begin{tabular}{l}
%Петрова Надежда Александровна\\
%Инженер-исследователь\\
%ОАО Компания "Нейросети и деревья" 
%\end{tabular}}

% консультант (если есть)
%\vspace{1em}
%{\bf Консультант: \vspace{2mm}}%это ваш официальный научник

%{\setstretch{1.1}
%\begin{tabular}{l}
%Иванова Надежда Александровна\\
%Инженер-исследователь\\
%ОАО Компания "Нейросети и деревья" 
%\end{tabular}}

\vspace{\fill}

\begin{center}
Москва 2024
\end{center}

\end{titlepage}% это титульный лист - выберите подходящий вам из имеющихся в проекте вариантов (kr - курсовая работа у 3 курса, vkr - выпускная квалификационная работа у 4 курса)
\newpage
\setcounter{page}{2}

{   

	\hypersetup{linkcolor=black}
	\tableofcontents
}

\newpage

\newpage
\section*{Аннотация}   % this is how to use russian

\addcontentsline{toc}{section}{Аннотация}
Алгоритмы распределенной оптимизации становятся все более актуальными,
так как современные модели машинного обучения зачастую тренируются на массивах данных,
которые невозможно разместить на одном вычислительном устройстве в силу их большого размера или приватности.
В данной работе рассматривается случай централизованной оптимизации, где один сервер хранит параметры модели,
данные разделены между несколькими устройствами и сервером, управляющим процессом оптимизации.
Целью работы является обобщение оценок полученных в случае сильно выпуклых функций на более широкий класс функций
в условиях схожести слагаемых. 
\section*{Ключевые слова}
Методы оптимизации, распределенная оптимизация, централизованная оптимизация, условие Поляка-Лоясиевича, схожесть слагаемых
\pagebreak


% chktex-file 8
% chktex-file 3
% chktex-file 25

\section{Введение}

\subsection{Описание предметной области}
\newcommand{\zij}{z_{i}^{(j)}}
\newcommand{\lxz}{l (x, \zij)}
\newcommand{\fj}{\sum_{i=1}^{n} \lxz}
\newcommand{\R}{\mathbb{R}}

Распределенная оптимизация возникает в случае, когда функционал зависит от данных, размещенных на различных носителях.
Например, в случае поиска параметров некой модели машинного обучения оптимизируется эмпирическая функция потерь.
В централизованном случае распределенной оптимизации данные распределены между сервером и другими компьютерами.
Происходит итерационный процесс в ходе которого сервер поручает другим компьютерам выполнять некоторые вычисления,
они направляют результаты обратно серверу и сервер выполняет шаги оптимизационного процесса, используя данные,
размещенные на нем и полученные значения от других компьютеров


\subsection{Постановка задачи}
В данной работе работе рассматривается централизованная оптимизация в случае, когда слагаемые,
из которых состоит оптимизируемый функционал, в какой-то мере схожи между собой.
Целью работы является доказательство оценки на сходимость уже известных алгоритмов в терминах условия Поляка-Лоясиевича
и исследование возможностей улучшения алгоритмов с учетом ограничений на функционал. 


\section{Формальная постановка задачи}

Скажем, что всего есть $m$ компьютеров и $n$ элементов обучающей выборки на каждом, $N = nm$ -- суммарное количество примеров.
Обозначим за $z_i^{(j)}$ -- элемент обучающей выборки под номером $i$ на машине номер $j$. Будем обозначать за $x\in\R^d$ параметры.
Обозначим за  $l(x, z)$ -- функцию потерь на элементе $z$, при векторе параметров $x$. Кроме того введем следуюшие обозначения:

\begin{equation}
    \label{eq:f_i}
         f_i(x) = \frac{1}{n} \sum_{j=1}^{n} l(x, z_i^j)
\end{equation}

\begin{equation}
    \label{eq:F}
         F(x) = \frac{1}{m} \sum_{i=1}^{m} f_i(x)
\end{equation}

То есть $F(x)$ представляет из себя эмпирический риск на всем массиве данных, а $f_i$ -- на компьютере $i$.
Тогда нашей целью будет являться решение следующей задачи оптимизации
\begin{equation}
    \label{eq:opt_x}
         \min_{x\in\R^d}~~ F(x)+\psi(x)
\end{equation}


Где $\psi(x)$ представляет из себя некоторую функцию регуляризации. Для начала положим $\psi(x) = 0$. 



\section{Обзор литературы}
\subsection{Условие сильной выпуклости}

Функция $f:\R^d\to\R$ называется сильно выпуклой на множестве $S$, если существует такая положительная константа $\mu$, что 
\begin{equation}\label{eqn:strong_convex}   
    \mu I \preceq \nabla^2 f(x), \quad \forall\, x\in S
\end{equation}

Это условие часто выполняется для изучаемых нами просто выпуклых функций из-за добавления $L_2$ регуляризации.
В статье (Hendrikx et al., 2020)\cite{Hendrikx-paper} с его учетом дается оценка скорости сходимости,
которую мы приведем после описания еще одного важного условия. 

\subsection{Условие схожести слагаемых}
 Скажем, что функции $f(x)$ и $g(x)$ являются $\delta$-cхожими на некотором множестве $S$, если выполнено следующее. 
 \begin{equation}\label{eqn:Hessian-approx-mu}
    \left\|\nabla^2 f(x) - \nabla^2 g(x)\right\| \leq \delta, 
    \quad \forall\, x\in S,
\end{equation}
Рассматриваемая нами функция $F$ имеет вид суммы нескольких слагаемых.
Заметим, что если данные на каждом компьютере получены случайным выбором без возвращения $n$ элементов из обучающей выборки,
то $f_i$ весьма вероятно похожи между собой и хорошо приближают функцию $F$.
А именно, в (Troop J., 2015)\cite{Troop-paper} с помощью матричного неравенства Хеффдинга показано,
что с вероятностью не меньше $1-p$ выполнено:

\begin{equation}\label{eqn:matrix-hoeffding}
    \biggl\|\nabla^2f_i(x)\!-\!\nabla^2 F(x)\biggr\|
    \!\leq\! \sqrt{\frac{32 A_l^2 \log(d/p)}{n}},
\end{equation}
Для некоторого числа $A_l$ такого, что $A_l \geq \|\nabla^2\ell(x,z_i)\| \ \forall\, x\in S$.


Использовать $\delta$-cхожесть для уменьшения числа итераций алгоритма в случае сильно выпуклых функций можно,
если воспользоваться зеркальным спуском c дивергенцией Брэгмана $D_{\phi}(x, x_t)$, определяемой следующим образом


\begin{equation}\label{eqn:Bregman-def}
    D_\phi(x, y) \triangleq \phi(x) - \phi(y) - \nabla \phi(y)^\top(x - y).
\end{equation}


Скажем, что сервер имеет номер 1 среди всех машин, тогда введем следующую $\phi(x)$
\begin{equation}\label{eqn:reference-def}
    \phi(x) = f_1(x) + \frac{\delta}{2}|x\|^2.
\end{equation}



Теперь можно модифицировать предыдущий алгоритм.

\begin{algorithm}[]
\caption{Зеркальный спуск в условиях схожести слагаемых}
\begin{algorithmic}[1]

\STATE{} receive $(\eta, x_1, K)$ as input 
\vspace{0.5ex}
\FOR{$t=1,2\ldots, K$} 
\vspace{0.5ex}
\STATE{} send $x_t$ to all computers 
\vspace{0.5ex}
\vspace{0.5ex}
\FOR{$i=1,2\ldots, m$} 
\vspace{0.5ex}
\STATE{} receive $x_t$
\vspace{0.5ex}
\STATE{} compute $\nabla f_i(x_t)$
\vspace{0.5ex}
\STATE{} send $\nabla f_i(x_t)$
\vspace{0.5ex}
\ENDFOR{}
\vspace{0.5ex}
\STATE{} receive $\nabla f_1(x_t), \nabla f_2(x_t), \ldots, \nabla f_m(x_t)$
\vspace{0.5ex}

\STATE{} compute $\nabla F(x_k)= \frac{1}{m} \sum_{i = 1}^{m}  \nabla f_i(x_t)$\
\vspace{0.5ex}

\STATE{} $x_{t+1} = \argmin_{x\in \R^d} \Bigl\{\nabla F(x_t){}^\top\! x + \frac{1}{\eta} D_\phi(x,x_{t}) \Bigr\}$.
\vspace{0.5ex}

\ENDFOR{}
\RETURN{} $x_{K}$
\end{algorithmic}
\end{algorithm}



Для него выполняется оценка (Hendrikx et al., 2020)\cite{Hendrikx-paper}:
\begin{equation} \label{eqn:mirror_convergence}
    D_\phi(x*, x_{k+1}) \leq \left (1 - \frac{\mu}{\mu + 2\delta} \right ) ^k D_\phi(x*, x_1)
\end{equation}

Где $x^* = \argmin F(x)$.


\section{Формальная постановка задачи}
    \[ r = \sum_{i=1}^n f_i \]
    \[ r = p + q \]
    \[q = f_1 \] 
    \[p = \frac1n \sum \left[ f_i - f_1\right] \]
\section{Методы решения задачи}
\subsection{Примитивы}
Для начала определим два примитива, которыми будут пользоваться последующие алгоритмы

    \begin{algorithm}[]
    \caption{Подсчет $\nabla r(x)$}
    \begin{algorithmic}[1]
    
    \STATE{} \textbf{Input}: $x \in \mathbb{R}^d$
    \STATE{} send $x$ to all computers 
    \vspace{0.5ex}
    \vspace{0.5ex}
    \FOR{$i=1,2\ldots, n$} 
    \vspace{0.5ex}
    \STATE{} receive $x$
    \vspace{0.5ex}
    \STATE{} compute $\nabla f_i(x)$
    \vspace{0.5ex}
    \STATE{} send $\nabla f_i(x)$
    \vspace{0.5ex}
    \ENDFOR{}
    \vspace{0.5ex}
    \STATE{} receive $\nabla f_1(x_t), \nabla f_2(x_t), \ldots, \nabla f_n(x_t)$
    \vspace{0.5ex}
    
    \STATE{} compute $\nabla r(x)= \frac{1}{n} \sum_{i = 1}^{n}  \nabla f_i(x_t)$\
    \RETURN{} $\nabla r(x)$
    \end{algorithmic}
    \end{algorithm}
    Приведем подробное описание принципа работы,
    в дальнейшем в аналогичном контексте используется такая же схема
    взаимодействия сервера и других компьютеров

    Сервер получает на вход значение $x$. В строчке 2 он отсылает его всем компьютерам. 
    После чего все компьютеры параллельно исполняют строчки 4, 5 и 6.
    В строчке 8 сервер получает резульаты вычислений всех компьютеров. 
    В конце он вычисляет $\nabla r(x)$, используя полученные результаты, и возвращает результат. 


    Данное действие требует 1 раунд коммуникаций и $n$ локальных вызовов оракула градиента. 


    \begin{algorithm}[]
    \caption{Подсчет $\nabla q(x)$}
    \begin{algorithmic}[1]
    
  
    \STATE{} send $x$ to all computers 
    \vspace{0.5ex}
    \vspace{0.5ex}
    \FOR{$i=2,3\ldots, n$} 
    \vspace{0.5ex}
    \STATE{} receive $x$
    \vspace{0.5ex}
    \STATE{} compute $\nabla f_i(x)$
    \vspace{0.5ex}
    \STATE{} send $\nabla f_i(x)$
    \vspace{0.5ex}
    \ENDFOR{}
    \vspace{0.5ex}
    \STATE{} receive $\nabla f_2(x_t), \nabla f_2(x_t), \ldots, \nabla f_n(x_t)$
    \STATE{} compute $\nabla f_1(x)$
    \STATE{} compute $\nabla q(x) = \frac1n \sum_{i = 1}^{n} \left[\nabla f_i(x)  - \nabla f_1(x) \right]$
    \vspace{0.5ex}
    \RETURN{} $\nabla q(x)$
    \end{algorithmic}
    \end{algorithm}

    Этот алгоритм использует схему вычислений, аналогичную предыдущей,
    и также требует 1 раунд коммуникаций и $n$ локальных вызовов оракула градиента. 



\subsection{Централизованная версия градиентного спуска}

Приведем пример работы градиентного спуска в условиях централизованной оптимизации.

\begin{algorithm}[]
    \caption{Централизованный градиентный спуск}
    \begin{algorithmic}[1]

    \STATE{} \textbf{Input}: $x^0 \in \mathbb{R}^d$
    \STATE{} \textbf{Parameters}: $\eta, K$
    \vspace{0.5ex}
    \FOR{$t=0,1\ldots, K - 1$} 
    \vspace{0.5ex}
    \STATE{} $x_{t + 1} = x_t - \eta \nabla r(x_t)$
    \vspace{0.5ex}
    \ENDFOR{}
    \RETURN{} $x_{K}$
    \end{algorithmic}
\end{algorithm}

Где $\eta = \frac{1}{L}$

\subsection{Централизованная версия ускоренного градиентного спуска}

\begin{algorithm}[]
    \caption{Централизованный ускоренный градиентный спуск}
    \begin{algorithmic}[1]

    \STATE{} \textbf{Input}: $x^0 \in \mathbb{R}^d$
    \STATE{} \textbf{Parameters}: $\eta, \kappa, \mu, L, \gamma$
    \STATE{} $y_0 = x_0$
    \vspace{0.5ex}
    \FOR{$t=0,1\ldots, K - 1$} 
    \vspace{0.5ex}
    \STATE{} $y_{t + 1} = x_{t} - \eta \nabla r(x_t)$
    \STATE{} $x_{t + 1} = (1 + \gamma)y_{t+1} - \gamma y_{t}$
    \vspace{0.5ex}
    \ENDFOR{}
    \RETURN{} $x_{K}$
    \end{algorithmic}
\end{algorithm}

Где $\eta = \frac{1}{\eta}$, $\kappa = \frac{L}{\mu}$, $\gamma = \frac{\sqrt{\kappa} - 1}{\sqrt{\kappa} + 1}$

Данные алгоритмы эквивалентны градиентному и ускоренному градиентному спуску в
классическом нераспределенном случае по производимым вычислениям.
Мы будем их использовать в качестве базовых решений для сравнения эффективности по числу коммуникаций и
числу вызовов оракула градиента $f_i$.


\subsection{Accelerated Extragradient}

В статье [???] приведен алгоритм,
который является асимптотически оптимальным как по числу локальных вызовов оракула градиента,
так и по числу коммуникаций. 


\begin{algorithm}[]
    \caption{Accelerated Extragradient}
    \begin{algorithmic}[1]

    \STATE{} \textbf{Input}: $x^0 \in \mathbb{R}^d$
    \STATE{} \textbf{Parameters}: $K, \eta, \theta, \alpha$
    \STATE{} $x_f^0 = x^0$
    \FOR{$t=0,1\ldots, K - 1$} 
    \STATE{} $x_g^k = \tau x^k + (1-\tau)x^k_f$
    \STATE{} $x_f^{k+1} \approx \argmin_{x \in \R^d}\left[ A_\theta^k(x) \eqdef p(x_g^k) + \<\nabla p(x_g^k),x - x_g^k> + \frac{1}{2\theta}\sqn{x - x_g^k} + q(x)\right]$
    \STATE{} $x^{t+1} = x^t + \eta\alpha (x_f^{t+1}  - x^t)- \eta \nabla r(x_f^{t+1})$
    \ENDFOR{}
    \RETURN{} $x^{K}$
    \end{algorithmic}
\end{algorithm}

В данном случае в строчке 6 имеется в виду приближенное решение задачи, для которого соблюдается условие 
\[	\sqn{\nabla A_\theta^k(x_f^{k+1})} \leq  \frac{L_p^2}{3}\sqn{x_g^k- \argmin_{x \in \R^d} A_\theta^k(x)}\]

\subsubsection{Алгоритм решения}

\subsection{Modified Accelerated Extragradient}
Заметим, что условие [???] непрактично, так как невозможно применять итерационный метод 
для поиска решения и прекращать итерации, как только оно стало выполнено. Это из-за того 
что в выражении справа необходимо знать $\argmin_{x \in \R^d} A_\theta^k(x)$. Поэтому 
в представленном в статье [???] алгоритме необходимо делать фиксированное число итераций 
оптмизационного процесса такое, что условие [] гарантированно выполнено. 

В нашем методе будем искать решение такое, что выполняется другое условие:
\[	\sqn{\nabla A_\theta^k(x_f^{k+1})} \leq  L_p^2 \sqn{x_g^k- x_f^{k+1}} \]
Искать эту точку будем, минимизируя $A_{\theta}^k$, с использованием ускоренного градиентного спуска,
начиная из точки $x_g$.



\begin{algorithm}[]
    \caption{Subproblem solver 2}
    \begin{algorithmic}[1]

    \STATE{} \textbf{Input}: $x_g \in \mathbb{R}^d$
    \STATE{} \textbf{Parameters}: $\eta, \kappa, \mu, L, \gamma$
    \STATE{} $y_0 = x_0$
    \vspace{0.5ex}
    \FOR{$t=0,1\ldots, K - 1$} 
    \vspace{0.5ex}
    \STATE{} $y_{t + 1} = x_{t} - \eta \nabla r(x_t)$
    \STATE{} $x_{t + 1} = (1 + \gamma)y_{t+1} - \gamma y_{t}$
    \vspace{0.5ex}
    \ENDFOR{}
    \RETURN{} $x_{K}$
    \end{algorithmic}
\end{algorithm}




\subsection{}

    В качестве начального решения будем использовать централизованную верси
\section{Теоретический анализ}
    \subsection{Оценка на количество итераций для решения подзадачи}

\section{Условие схожести слагаемых }
    оценить Lq через дельту
    $\delta = O (1/\sqrt{m})$

\section{Экспериментальное исследование метода}
    Будем рассматривать следующую задачу минимизации эмпирического риска (Ridge-регрессию)
    \[ F(w) = \frac{\lambda}{2} ||w||_2^2 + \frac{1}{2N} \sum_{i=1}^n (w^t x_i - y_i)^2 \to_{w} \min \]
     Где $x_i$ -- вектор параметров объекта $i$, $w$ -- параметры модели, $ \lambda $ -- коэффициент регуляризации.
    Эквивалентно ее можно переписать в матричном виде следующим образом: 
        \[\frac{1}{2N}||Xw - y||_2^2 + \frac{\lambda}{2} ||w||_2^2 \to_{w} \min \]
    Где $X$ -- матрица высоты $N$ и ширины $d$, в строке $i$ которой находится $x_i^t$.
    Будем ее называть матрицей дизайна. 
   
    Будем делать два типа экспериментов: на синтетических данных и реальных.
    В обоих случаях распределенность будет симулироваться.
    Мы будем производить все вычисления на одном устройстве, но при соверешении действий,
    эквивалентных коммуникации сервера и компьютеров, будем инкрементировать счетчик числа коммуникаций.
    Аналогично при подсчете градиента функции, относящихся к слагаемым на одной из машин,
    будем инкрементировать счетчик числа локальных вызовов. 

    \subsection{Эксперимент на синтетических данных}
    Будем рассматривать сеть из $n$ машин, первая из которых является сервером.
    На каждой машине будут располагаться $m$ примеров.
        
    \begin{itemize}
    \item  Сгенерируем матрицу дизайна для сервера $X_1$,
    где $X_{1ij} \sim^{i.i.d} \mathcal{N}(0, L)$ c использованием  \textbf{numpy}.
    Константа L позволяет регулировать константу липшица градиента функции потерь по параметрам.
    Чем больше L, тем в среднем выше константа липшицевости. 
    
    \item После чего сгенерируем вектор $w$, $ w_i \sim^{i.i.d} \mathcal{N}(0, 1)$. 
    
    \item Теперь положим $y = X_1w + \varepsilon$, где $\varepsilon_i \sim^{i.i.d} \mathcal{N}(0, 1)$
    \newcommand\tab[1][1cm]{\hspace*{#1}}
    

    \tab{} Это соответствует классическим статистическим предположениям в задаче линейной регрессии. 

    \item Сгенерируем матрицы дизайна других машин ($k \geq 2$) следующим образом $X_k = X_1 + \delta P_k$,
     где $P_{kij} \sim^{i.i.d} \mathcal{N}(0, 1)$.
    Константа $\delta$ позволяет регулировать схожесть данных на сервере и других машинах.
    Чем она меньше, тем в среднем ближе будут гессианы функций на разных машинах. 
    
    \end{itemize}

    Частично для приближения к условиям статьи [] были выбраны следующие константы 
    \begin{itemize}
        \item $\lambda = 0.1$
        \item $n = 25$
        \item $m = 100$
        \item $d = 50$
        \item $\delta = 0.1$
        \item $L = 2$
    \end{itemize}

    \subsection{Эксперимент на реальных данных}
    В качестве реальных данных для эксперимента использовался
    \href{https://www.kaggle.com/datasets/budincsevity/szeged-weather/data}{набор данных},
    в котором собрана информация о погоде. 
 
    В качестве признаков выбирались столбцы 
    \begin{itemize}
        \item ``Wind Speed (km/h)'
        \item ``Humidity'
        \item ``Wind Bearing (degrees)'
        \item ``Visibility (km)'
        \item ``Loud Cover'
        \item ``Pressure (millibars)'
        \item ``Temperature (C)'
    \end{itemize}
    В качестве предсказываемой величины ``Apparent Temperature (C)'.

    
    Кроме того к признакам добавлен единичный свободный член.
    Признаки отнормированы так, что имеют нулевое среднее и единичную выборочную дисперсию.
    Объекты распределены между 25 машинами в случайном порядке. 

    
    \subsection{Оценка констант $\delta$, $L_p$, $L_q$, $\mu$}

    Для работы некоторых методов необходимо знание констант гладкости и сильной выпуклости.
    Кроме того их знание полезно для отслеживания зависимостей качества различных методов от свойств данных. 
     \[ f_i(w) = \frac{1}{m}||X_i w - y||_2^2 + \frac{\lambda}{2} ||w||_2^2\]
    \[ \nabla_{w} f_i = \frac{1}{m}X_i^T(X_i w - y) + \lambda w\]
    \[\nabla_{w}^2 f_i  = \frac{1}{m}X_i^T X_i + \lambda I_d\]

    \subsubsection{Оценка $L_q$}
   
    \[q(w) = f_1(w) \]
    \[\nabla_{w}^2 q =  \frac{1}{m}X_i^T X_i + \lambda I_d\]
    \[L_q =  \frac{1}{m} \lambda_{\max}(X_i^T X_i ) + \lambda\]

    \subsubsection{Оценка $L_p$}
    \[p(w) = \frac{1}{n} \sum_{i=1}^{n} \left[f_i(w) - f_1(w)\right]\]
    \[\nabla_{w}^2p = \frac{1}{n}\sum_{i = 1}^{n} [\nabla_{w}^2f_i - \nabla_{w}^2f_1] = \]
    \[ = \frac{1}{n}\sum_{i = 1}^{n} [\frac{1}{m}X_i^T X_i + \lambda I_d - \frac{1}{m}X_1^T X_1- \lambda I_d] = \] 
    
    \[=\frac{1}{n}\sum_{i = 1}^{n} [\frac{1}{m}X_i^T X_i]- \frac{1}{m} X_1^T X_1 = \]

    \[L_p = \lambda_{\max}(\frac{1}{n}\sum_{i = 1}^{n} [\frac{1}{m}X_i^T X_i]- \frac{1}{m} X_1^T X_1) \]

    Где $\lambda_{\max}$ -- наибольшее собственное значение матрицы,
    которое можно найти с помощью функции \textbf{eigvalsh} из пакета \textbf{scipy},
    которая подходит для работы с симметричными вещественными матрицами. 
     

    
    \subsubsection{Оценка $\delta$}
    Отметим, что в нашем случае у всех функций гессиан является константным, поэтому 
    \[\delta = \max_{i} ||\nabla_{w}^2 f_1 - \nabla_{w}^2 f_i||_2 = \frac{1}{m} \max_{i} ||X_1^T X_1 - X_i^T X_i||\]
    
    \subsubsection{Оценка $\mu$}
    Каждая функция $\frac{1}{m}||X_iw - y||_2^2$ является выпуклой,
    $\frac{\lambda}{2} ||w||_2^2$ является $\lambda$-сильно выпуклой,
    поэтому  каждая функция $f_i$ и $r = \sum\frac{1}{n} f_i$ являются $\mu$-сильно выпуклой,
    где $\mu = \lambda$.

    \subsection{Оценка сходимости методов}
    Для оценки сходимости каждого метода на итерации $T$ будем вычислять во сколько раз квадрат расстояния до оптимума меньше,
    то есть 
    \[\frac{||w^T - w_*||_2^2}{||w^0 - w_*||},\]
    где 
    \[w_* = \argmin \frac{1}{2N}||Xw - y||_2^2 + \frac{\lambda}{2}||w||_2^2\]
    Во всех методах будем полагать $w_0 = 0$.
    Для задачи (todo) сущетсвует точное аналитическое решение:
    \[w_* = (X^T X + N\lambda I_d)^{-1}X^T y,\]
    которое мы будем использовать в экспериментах. 
    
    Отметим, что зачастую аналитическое решение непрактично из-за высокой временной сложности
    обращения матриц и вычислительной неустойчивости, поэтому на практике зачастую используются именно итерационные методы,
    подобные методам, описываемым в данной работе. Кроме того для широкого класса задач,
    не существует явного аналитического выражения решения, поэтому такие методы 
    
\newpage 
\printbibliography[heading=bibintoc] 
	
\end{document}
